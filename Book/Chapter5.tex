\section{Заключение}
\label{sec:Chapter5} \index{Chapter5}
В данной работе были исследованы методы OSINT для поиска информации о человеке. Её решение было разбито на следующие задачи:
\begin{itemize}
    \item Поиск данных в поисковых сервисах:
    \begin{itemize}
        \item Провести анализ литературы и существующих решений для извлечения данных из поисковых систем;
        \item Разработать методы поиска и сбора информации из поисковых систем.
    \end{itemize}
    \itemПоиск данных в социальных сетях:
    \begin{itemize}
        \itemПровести анализ литературы и существующих решений для извлечения данных из социальных сетей;
        \itemРазработать методы поиска и сбора информации из социальных сетей.
    \end{itemize}
\end{itemize}

В рамках решения описанных выше задач были решены следующие:
\begin{itemize}
    \item проведен обзор литературы, статей, посвященных описанию различных OSINT-методов. Обзор показал, что методы поиска и 
    сбора могут отличаться, самые передовые приложения самостоятельно ищут, собирают и анализируют данные, представляю их 
    далее в виде дерева зависимостей;
    \item проведен обзор литературы, статей, связанных с устройством фреймворка Scrapy, Splash. Обзор показал, что на данный 
    момент использование Scrapy полностью оправдано, если необходимо производить сбор обширных данных на протяжении большого 
    количества времени, так и доказал, что использование Splash для рендера html-страниц полностью оправдано в нашей системе;
    \item Разработы методы поиска и сбора информации, собраны тестовые данные для оценки их качества;
    \item Проведена оценка качества с помощью сторонних метрик, которые отображают частоту вхождений каждого из полей.
\end{itemize}

По результату исследований разработаны и реализованы OSINT-методы поиска и сбора данных из поисковых источников и социальной сети
LinkedIn на языке Python.
\par
Данная работа может быть продолжена в следующих направлениях:
\begin{itemize}
    \item проведение исследований о возможности включения большего количества поисковых ресурсов и социальных сетей в систему;
    \item проведение исследований о возможности построения деревьев связи и наглядного отображения зависимостей в пользовательском
    интерфейсе.
\end{itemize}

Следует отметить, что разрабатываемое решение для LinkedIn осуществляет переходы только по URL-ссылкам, находящимся в атрибутах.
Но современные сайты могут использовать динамическую подгрузку данных, и на этот случай реализован рендер страницы при помощи 
Splash, а так же выявлен способ извелечения данных через API. Так что можно сделать вывод, что на данный момент система является
самодостаточной: она может обрабтывать статические, динамические сайты и вызовы через API.