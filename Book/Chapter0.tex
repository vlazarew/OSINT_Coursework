\section{Введение}
\label{sec:Chapter0} \index{Chapter0}
В современном мире присутствует огромное количество социальных сетей и поисковых ресурсов, которые имеют собственные стараницы в 
сети Интернет. Это могут быть различные социальные сети: от медиа (Instagram, TikTok), мессенджеров (Telegram, WhatsApp), так
и полноценных, в которых можно указывать информацию о личности (ВКонтакте, Facebook, LinkedIn). И большинство людей имеют аккаунты
сразу в нескольких социальных сетях одновременно, самостоятельно и по доброй воле делятся своими персональными данными.
\par
Вместе с этим активно развиваются сервисы, которые делают подборку контента на основе агрегации и обработки данных, полученных
из Интернет. Даже та самая контекстная реклама, которая старается продвинуть товары и услуги, которые недавно искались пользователем -- 
есть часть тех самых сервисом и OSINT в целом \cite{yushchuk}. Например YouTube отображает в рекомендованных видеозаписях тот контент,
которых находится на стыке популярного сейчас и тех тематик, которые просматривали ранее. Также существуют сервисы по предоставлению
персонализированных новостных лент, самой популярной в RU-сегменте является Яндекс.Дзен. У этого подхода есть существенные плюсы, 
такие как пользователь всегда будет актуальный и необходимый ему контент.
\par
Говоря дальше об OSINT, стоит упомянуть, что точный термин ставится как <<разведка на основе открытых источников>>. То есть, в 
сборе и обработки данных нет ничего противозаконного, так как никакие базы данных и устройства не взламываются. Но и этого количества
информации весьма достаточно, чтоб иметь некую картину о пользователе сети Интернет или организации с активной социальной жизнью.
С помощью данной технологии правительства всех стран могут отслеживать и поддерживать национальную безопасность, бороться с 
терроризмом и устанавливать слежку за участниками преступных группировок, оценивать настроения и взгляды общественности как внутри
государства, так и вне ее.
\par
Например, такое ПО как Palantir активно используется в полиции для отслеживания преступников, ведь оно агрегирует данные не только из Интернет,
но и предоставляет картинку с камер видеонаблюдения и строит зависимости на географической карте страны.
\par
Таким образом, появляется задача разработки обширной и автоматической системы поиска и сбора данных из открытых источников сети Интернет, 
способных извлекать данные установленного формата из большого количества веб-ресурсов. Под обширностью понимается, что необходимо
задействовать по максимуму все возможные поисковые ресурсы, ведь именно в них данные уже заранее проанализированы и структурированы.
Под автоматизацией подразумевается то, что оператору системы необходимо будет только единожды настроить параметры сбора, а информация
будет автоматически далее собираться, отсеивать дубликаты и сохраняться в базу данных.


\par
В разделе 1 сформулирована постановка задачи. 
В разделе 2 приведен анализ сущестующих решений методов поиска, сбора и анализа информации из открытых источников. 
В разделе 3 описано исследование и построение решения задачи. 
В разделе 4 приведено описание практической части курсовой работы. 
В конце документа сформулировано заключение.
