\section{Постановка задачи}
\label{sec:Chapter1} \index{Chapter1}
Целью данной курсовой работы является исследование и разработка методов OSINT для поиска информации о человеке. 
Для решения задачи, ее можно разбить на несколько подзадач: сбор информации при помощи поисковых сервисов, 
сбор информации с помощью социальных сетей. В свою очередь каждую из подзадач также можно поделить на следующие части:
определение структуры web-страницы и извлечение данных непосредственно из страницы, поиск более быстрого доступа к информации 
посредством открытого или закрытого api.
\parВ итоге для достижения постановленной цели необоходимо решить следующие задачи:
\begin{itemize}
    \itemПоиск данных в поисковых сервисах:
    \begin{itemize}
        \itemПровести анализ литературы и существующих решений для извлечения данных из поисковых систем;
        \itemРазработать методы поиска и сбора информации из поисковых систем:
        \begin{itemize}
            \itemПроанализировать структуру web-страниц поискового сервиса;
            \itemРеализовать метод поиска и извлечения информации при помощи атрибутов web-страницы;
            \itemПровести исследование о возможности получения данных из ресурса посредством открытого или закрытого api;
            \itemЕсли api реализовано на стороне сервиса, то реализовать метод поиска и сбора посредством api;
        \end{itemize}
        \itemПолучить тестовые данные от реализованных методов и провести анализ, исследование полученной информации; 
    \end{itemize}
    \itemПоиск данных в социальных сетях:
    \begin{itemize}
        \itemПровести анализ литературы и существующих решений для извлечения данных из социальных сетей;
        \itemРазработать методы поиска и сбора информации из социальных сетей:
        \begin{itemize}
            \itemПроанализировать структуру web-страниц социальных сетей;
            \itemРеализовать метод поиска и извлечения информации при помощи атрибутов web-страницы;
            \itemПровести исследование о возможности получения данных из ресурса посредством открытого или закрытого api;
            \itemЕсли api реализовано на стороне соц. сети, то реализовать метод поиска и сбора посредством api;
        \end{itemize}
        \itemПолучить тестовые данные от реализованных методов и провести анализ, исследование полученной информации; 
    \end{itemize}
\end{itemize}